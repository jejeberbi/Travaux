\documentclass{book}

\usepackage[latin1]{inputenc}
\usepackage[T1]{fontenc}
\usepackage[francais]{babel}
\usepackage[top=2cm, bottom=2cm, left=4cm, right=4cm]{geometry}
\usepackage{url}
%\usepackage{charter}
\pagestyle{plain}
\author{Jejeberbi : www.jejeberbi.eu}
\title{Chronique d'une guerrière communiste \newline Titre provisoire}

\begin{document}

\maketitle
\tableofcontents

\part{Changement de vie}

\chapter{Une journée pas comme les autres}
Comme tous les jours, j'arrivai la première. Le temps était calme, le ciel bleu, quelques nuages. Routine. Je restais assise là, à attendre. Ennui. Je regardais devant moi, je fixai le paysage, j'attendais. Encore. Encore. \newline

Quand je me demandai ce que nous allions faire de cette nouvelle journée, un de mes compagnons arriva. Enfin ! "Coucou" me dit-il sur un ton des plus abrutis qu'il pouvait prendre. Puis un second arriva "Ha que coucou !" dit il sur un ton rauque. Il en manquait un, un seul. Il arriva enfin après que nous nous soyons échangés les formules de politesse de base. Il dit bonjour d'un simple "salut".\newline

L'équipe était là, que dis-je, la camaraderie, la compagnie était là ! Fini l'ennuie, fini de fixer un coin du paysage.. on allais pouvoir commencer.\newline

Nous nous étions réunis, comme à notre habitude, en bordure d'une vaste forêt surplombant les champs agricoles locaux. Le printemps pointait le bout de son nez, et tous les ouvriers s'étaient mis au travail. C'étais la saison des plantations, le travail le plus physique pour les paysans Krytiens. Ils devaient labourer les champs, pour accueillir les futurs semences qui deviendraient ensuite d'immenses champs de céréales d'ici le début de l'été. Ce sera le meilleur moment de l'année, le soleil levant, et ses premiers rayons reflétés par les étendus de céréales feront du paysage un immense miroir à une couleur or magnifique.\newline

Un de mes compagnons, alors très inspiré, nous proposa de faire comme d'habitude quand les travaux agricoles reprenaient. Tout le monde acquiesça. On se mit en route, on vadrouilla dans les bois environnant, jusqu'à arriver au prochain village voisin. En Kryte, les villages voisins ne pouvaient plus se côtoyer sereinement depuis que les centaures menaçaient les terres des humains. Tout le monde souhaitait la protection royale, et le moindre soldat séraphin assigné dans un village ou un autre entrainait jalousie et discorde entre deux coteries mitoyennes.\newline

Et nous, alors jeune que nous étions, on écoutait nos parents, au doigt et à la baguette, et notre activité préféré à cette période de l'année était de faire les 400 coups chez les voisins. Nous fabriquions quelques outils basique avec des branches, des cailloux.. Puis nous gâchions les champs, on faisait fuir le bétail, on utilisait les sources d'eau à notre avantage pour inonder les terres cultivables. Quand un paysan local nous remarquait, nous prenions la fuite à toutes jambes, en fou rire. Ce n'était pas évident de rire et courir en même temps, mais si nous étions pris, c'était directement chez les Séraphins, et eux, ils ne riraient pas, loin de là.\newline

La journée terminé, on rentra tous chez nous, on allais recommencer dès le lendemain. C'étais sans compter la fin de journée...\newline

Une fois rentré au village, on se dispersa chacun dans notre coin, pour rentrer dans nos familles respectives. Ma maison était en dehors du village, j'habitai le moulin à eau du village, une vétuste demeure qui avait plutôt pour proximité les champs que les autres habitants. Cela faisait un peu de marche pour aller au puits du village, mais on y était tranquille.\newline

La nuit tombé, j'allais vers ma chambre, direction le lit, où j'allais pouvoir me reposer de toutes ces aventures. A force de vivre comme ça, j'avais acquis une certaine force physique et physiologique, je n'étais même pas fatigué. Il faudrait peut être qu'on vise plus gros, plus risqué, plus marrant. Un vrai chalenge quoi !\newline

Quand à ma famille, comme à son habitude, elle retournera au village, pour aller boire un coup avec les amis, discuter moisson, et se plaindre une énième fois que la reine fais mal son devoir envers les citoyens. Ils finiraient tous pas rentrés fin ivre de toutes façon, en ayant répété leurs discours de la veille.\newline

Perdu dans mes pensées pendant un long moment, je me redressa vivement quand j'aperçus une vive lumière orange à travers ma fenêtre. Je regarda, un halo orange brillait au loin, en direction du village. Quand je redressa ma tête pour regarder en direction du ciel, qui jusque là était d'une beauté sans précédente, avec une voie lacté s'étendant vers les infinis et semblable à des abysses, je m'aperçus qu'une épaisse fumée nacre recouvrait ce ciel autrefois magnifique.\newline

Peur.\newline

J'ouvris la fenêtre, des cris stridents vinrent jusqu'à moi, le tout accompagné de bruit de sabots, en rythme tel des tambours de guerre. J'étais terrifiée. Je me leva, courut dans la maison, je voulais voir le reste de ma famille, ils n'étaient pas rentrés.\newline

Hésitation.\newline

Que faire ? Comment ? Je voulais les sauver, mais je n'avais rien pour me défendre. Si les Séraphins en poste ici n'avaient pu les arrêter, comment moi, une fille de paysans ayant tout juste atteint sa majorité aurait pu les arrêter ? Et sans armes en plus !\newline

Panique.\newline

Je me dit que si il doit y avoir des survivants, je les retrouverai plus tard, je pris la décision qui pour moi semblai la plus logique, je courra vers la porte, l'ouvrit d'un coup d'épaule et je continua de courir dans le sens opposé au village. Je ne savais où aller.\newline

Regret.\newline

J'aurai pu mourir au beau milieu d'une nature sauvage et dangereuse, ça aurait été pareil que de foncer en héroïne sans arme sur les ennemis. Je me demanda l'espace d'un instant si j'avais fais le bon choix, mais une explosion interrompit ma réflexion, c'était sans doute le ravitaillement des Séraphins qui venait d'exploser, ils entreposaient armes, armures, et bien sûr poudre à canon.\newline

Je continua donc de courir, encore, encore. Plus je courrais, plus ma vie s'éloignait, plus mes amis s'éloignaient, plus ma famille s'éloignait. J'ai fui mes origines, mon lieu de vie, d'éducation, pour conserver ma vie, mais quelle vie ? Je n'ai plus rien. Mis à part peut être un destin. Alors je continua.\newline

Impatience.\newline

Je voulais savoir ce que le futur me réservai, à moi, à l'humanité. Je continua donc la fuite, je traversa d'abord la forêt qui surplombait notre village, descendit un versant de montagne, traversa une rivière, un torrent, je suivis ensuite un chemin, vers ce qui sembla être le nord. Je le sût car le soleil commençait à se lever, j'avais fuis toute la nuit. Je commençais à ralentir, je ne le voulais pas, mais je n'eut pas le choix.\newline

Inconscience.

\chapter{Le Promontoire Divin}
Je sentis le vent soulever fébrilement les quelques jutes de lins abimées dans ma course effrénée qui me servait de vêtement. J'avais mal. Mal au crâne, mais surtout aux mollets sur lesquels j'avais trop poussé pendant ma fuite. Que m'était-il arrivé ? Je n'avais même pas la force de me poser la question, je sentais juste quelques vibrations et un sifflement si strident que j'avais l'impression qu'un centaure avais explosé ma tête de sa masse.\newline

Le vent étais frais, doux. Il était bon, relaxant. On aurait dis une brise d'un matin d'été, ce genre de brise agréable comparable à une main de bébé qui nous caresse le long du corps, remontant doucement les hanches jusqu'aux épaules pour s'envoler et recommencer de plus belle. J'avais envie de rester ainsi, mais peu à peu mes esprits revinrent. Le sifflement avait laissé place à un léger bourdonnement qui me laissait enfin penser.\newline

Il fallait que je sache ce qu'il m'était arrivé depuis que j'avais perdu connaissance. J'ouvris alors doucement les yeux, le soleil commençait à se lever. C'était bien le matin, tôt. Je ne distinguais rien, sauf la parole d'une petite fille qui étonné, dit avec entrain « Elle bouge, elle se réveille ! ». Elle parlait de moi.\newline

J'étais allongé dans une charrette, au milieu des cagettes de fruits et légumes entassés là, par un paysans à l'aurore sans doute. Je me raidis, tourna la tête à gauche, vit quelques arbres, qui tremblaient de plaisir avec le vent. Entre ceux ci, je distinguais un dénivelé haut d'une cinquantaine de mètre à pic, au pied duquel il y avait un immense cylindre de bois orné ci et là d'une grande plaque argenté.\newline

Je tourna la tête à droite. Au fond se dressait un immense mur couleur jaune mat sur lequel se reflétait les rayons du soleil levant. Je pus distinguer, malgré l'intensité de l'oeuvre de Melendru, quelques ouvertures en forme d'arche en son sein. A son pied coulait une petite étendue d'eau, qui, non régulière coulait jusqu'à moi, comme si la bénédiction de Lyssa venait me réconforter dans mon malheur.\newline

La charrette dans laquelle j'étais avais arrêter d'avancer. Nous étions stationné sur un pont, couvert d'un toit manufacturé en bois très sombre. Quelques banderoles tenaient ci et là quelques bougies, sans doute se servait-on de cette hauteur pour y attacher les décorations de fête.\newline

L'homme qui conduisait me fixait du regard, il avait un visage plutôt carré, de grand yeux marrons et une coupe ras, caché par son chapeau de paille. L'homme avais tout du paysans basique Krytien. Il était accompagné d'une fille, celle que j'avais entendu un peu plus tôt sans doute. Elle était caché derrière l'homme, laissant seulement dépasser sa tête, et surtout ses grands yeux bleu qui me regardaient comme si la chose la plus mystérieuse était à ma place. La petitesse de son visage, ses traits fins, et ses doux cheveux blonds soyeux laissaient facilement deviner qu'elle était jeune, très jeune, pas plus de sept ans.\newline

- Alors, ça vas mieux ? dit finalement l'homme sur un ton hésitant

Je ne sut que lui répondre. J'étais là, en face d'un inconnu, des heures après que j'eus échappé à une tragédie qui as réduit en cendre ma vie, et qui aurait bien pu m'expédier chez les dieux !

- Il ne me semble pas être chez les morts, dis-je finalement sur le ton le plus narcissique qu'il me soit possible de donner

Mes interlocuteurs ne savaient plus où se mettre. Qu'ai-je donc dis encore ? Je regrettais déjà mes paroles, ces gens m'avaient probablement sauvées, et je viens de montrer toute mon ingratitude. Le paysan ne savait plus où se mettre, il avait baissé les yeux. Je me repris finalement :

- Que m'est-il arrivé ? Où sommes nous ?\newline

Il me regarda, pris un peu d'assurance et me raconta qu'il me trouva sur la route qui le mène à son marché habituel. Le hameau où il vivait avait eu vent de l'attaque de mon village par les séraphins dès l'aurore. Il en avait déduit que j'étais une survivante. Il ne voulut pas me dire où j'étais, il me dit juste qu'il m'emmenait en sécurité.\newline

Je pris place aux côtés de la jeune fille du paysans. Elle continuait de me fixer, elle se posait des questions. Sans doute n'était elle pas au courant de la guerre avec les centaures, ses parents ne lui auraient rien dit car elle est trop jeune, ou peut être se demandait elle tout simplement ce que je faisais là. Moi même je me le demandais, j'étais gênée.\newline

La charrette repris sa course, traversa le pont sur lequel nous étions stationnés. Je me demandais ce que j'allais faire ensuite, comment j'allais m'en sortir, mais je n'avais pas la tête à réfléchir, alors j'observais. Devant nous, le chemin continuait au milieu d'un petit village, semblable au mien. A ma droite il y avait un petit abri, utilisé par les paysans locaux qui y entreposait leur matériel. Ici c'était l'abri pour le bois et tout le matériel de bucheron, à ma gauche s'élevait une grande maison, sans doute la plus grande du coin. Elle était de couleur rouge, brunit par l'épreuve du temps et des différentes saisons.\newline

Le chemin montait vers une structure non moins imposante, coloré, elle, de verte claire, qui aussi à cause du temps est sans doute passé d'un vert clair à un vert olive.\newline

Derrière, je distinguais comme une grande ombre, une immensité dont je ne pouvais distinguer la nature.\newline

Quand la voiture eut finis de traverser le pont, je distinguai que cette immensité s'étendait de ma gauche à ma droite. Cette immensité était construite de pierre massive et de mortier, qui reflétait la moindre lumière de Mélendru, cette vision semblait être une oeuvre de Lyssa, j'étais sidérée.\newline

Des grands arcs construits à plus de deux cent mètre de hauteur, reliaient une tour à une autre, chaque tour était plus grande que l'autre. En leur sommet, il y avait bon nombre de logements et mini maisons aménagés. Des structures carrés, ovales, rondes, jusqu'au formes les plus improbables ornaient les murs de cette cité. Derrière les murs, on distinguait encore d'autres chefs d'oeuvres qui montaient haut dans le ciel, à tel point que l'on pouvait confondre les monuments de la cité avec les nuages qui faisaient tranquillement leur vie à cette hauteur là !\newline

Je compris aussitôt pourquoi le paysan me dit qu'il m'emmenait en sécurité, j'étais là dans le coeur de la civilisation humaine, en son centre. J'arrivais dans une cité, vieille à peine d'un siècle, mais grande comme une éternité d'Histoire !\newline

\chapter{Le ministre Caudecus}
L'entrée de la ville était désormais derrière nous, le paysan m'expliqua alors l'organisation générale de la cité.\newline

La capitale était construite sur une forme de roue, préalablement pensé par les meilleurs architectes humains d'il y a de cela un siècle. De grandes avenues reliait deux axes circulaires, une extérieure longeant les murs, et une intérieure longeant les quartiers de l'administration : la ville haute. Ces boulevards étaient au nombre de six, portant chacun le nom d'un Dieu. L'entrée de la ville était sur la route de Dwayna, déesse de la vie... Ce paradoxe me fendait le coeur.\newline

J'écoutais distraitement les explications du brave homme, observer le prodige architectural de cette magnifique cité était bien plus intéressant. Le rustre m'expliqua alors la composition des différents quartiers, il y en avait un entre chaque route. Je n'eut pas le temps de me concentrer sur ce qu'il disait qu'on se retrouva arrivé à mon terminus. Il m'expliqua que c'était ici qu'on accueillait les réfugiés de la guerre qui avait tout perdu, et que c'était là que je trouverais toute l'aide dont j'ai besoin pour repartir d'un bon pied.\newline

Je ne répondis rien. Il finit son explication en me demandant comment je m'appelais.

- Jèssi, dis-je dans un marmonnement après un long moment de silence.

Il me regarda, fit le plus grand sourire qu'il put faire et me répondit.

- Bonne chance Jèssi\newline

Il repartit aussitôt. Je n'avais rien répondu. Ses mots sortaient du coeur, il était sincère. Sa fille continuait de me regarder, la charrette s'éloignant, elle me fit timidement un signe de la main. Pour la première fois depuis le drame, je sentis une lueur d'espoir naitre dans mon coeur. Ces deux personnes m'avaient sauvés la vie, je ne connaissais même pas leurs noms...\newline

On m'avais laissé près d'une place, dans un des quartiers populaires que mon sauveur avait surement dû évoquer. Il y avait quelques séraphins qui gardaient la zone. Mais cette petite agora était essentiellement bondé de réfugiés, dont beaucoup semblaient non seulement pauvre, mais aussi blessés, malades. A certains il manquait des membres, des pieds, des bras. Ça et là errait des bandages pleins de sang. Ceux qui étaient remplacés se déplaçaient de concert avec le vent.\newline

Des escaliers remontaient vers l'axe intérieur de la ville, en haut desquels la vue surplombait la place, je distinguais que bon nombre de ce qu'il semblait être des gens fortunés regardaient avec ardeur le spectacle qui s'offrait à eux. J'eus envie de les étriper. Après tout ce que j'avais vécu, je n'étais qu'un vulgaire numéro pour des gens qui ne savent rien de ce qu'il se passe en dehors de leurs murs.\newline

Je fis un trait sur ces misérables, il valait mieux ne pas faire attention à eux. J'observais avec attention les différents groupes répartis sur la place. D'un côté, on soignait les malades, d'un autre c'était les blessés ; à l'autre bout, était dressé un petit abri de toile blanc, monté à la hâte, sous lequel se dressait tables et bancs de fortune. On y distribuait un maigre repas. C'est ça dont j'avais besoin en ces temps troublés.\newline

Mangeant ce qui semblait être une soupe d'eau chaude aromatisé par pas plus de trois légumes, je réfléchis. Qu'allais-je faire maintenant ? Il me fallait un coin où dormir ; travailler, gagner de l'argent. Je n'avais envie de faire ces choses. A vrai dire, à ce moment précis, ma plus grande envie était de m'endormir au milieu de la foule et ne jamais me réveiller.\newline

Cependant, un homme vint à ma rencontre. Au vu de sa tête, sur laquelle il affichait un grand sourire satisfait, celui de l'homme qui as eu la meilleure idée du siècle, je m'inquiéta. Qu'allait-il me demander ?\newline

Il s'assit à mes côtés, me regarda fixement, et me salua poliment. Je fit de même, mais je fus plus froide, je n'étais pas là pour perdre mon temps. Finalement, après avoir cherché ses mots, il se présenta calmement. Il voulait gagner ma confiance, et me dis beaucoup de choses de lui. C'était un roturier, parcourant autrefois en Ascalon, et as été une des victimes des charrs avant le commencement des négociations de paix. Il s'était réfugié dans les plus petits hameaux Krytiens, mais ce fut ceux-ci qui cédaient les premiers face à l'invasion centaures. Il avait vécut plusieurs massacres, plusieurs batailles, tantôt victorieuses, tantôt désastreuse...\newline

Lui, c'était Yashu, il avait vécu la même chose que moi d'après ses dires. Voir même mile fois pires. Ça faisait un mois qu'il était là, il vivait d'action de mercenaires pour les habitants de Shaemoor, le village au pied de la cité. Il commençait à connaitre le moindre engrenage de cette cité.\newline

Quand on eut finit notre repas, il se leva, et me demanda de le suivre. Il voulait m'initier au fonctionnement local. Selon lui, quelque chose d'important allait avoir lieu. Je le suivit. Il monta les escaliers du haut desquels la populace en observe  une autre, il traça son chemin en suivant l'engrenage intérieur. Arrivé à l'embouchure de la route de Dwayna, il s'enfonça dans une ouverture où il monta de longs escaliers. J'étais fatigué, et j'avais toutes les peines du monde à le suivre.\newline

L'escalier déboucha sur un court corridor en arc de cercle, il y avait un sortie juste en face de la cage d'escalier, il l'emprunta. Je m'arrêtai brusquement. J'étais subjugué, nous étions dans la ville haute, le centre administratif, exécutif et névralgique de la puissance humaine. Au delà de la signification politique du lieu, c'était le décor qui était magnifique.\newline

Un grand dôme de verre s'élevait au dessus de moi, de nous. Il amplifiait la lumière du soleil et baignait les lieux dans une clarté bienveillante. Avec ses jardins aux plantes de toutes couleurs, s'élevant à quelques mètres de hauteur, le lieu semblait être le paradis sur la Tyrie. Le clou du spectacle était cette immense battisse juste en face de moi, qui était le palais Royal. L'immense porte couleur grenat, contrastait avec les verts jardins et la douce lumière qui émanait du dôme. Le lieu était tout simplement royal.\newline

Yashu se retourna vers moi, et rit. « On est jamais venu ici, hein ? » dit-il l'air moqueur. Il avait un sourire jusqu'aux oreilles. Je ne répondis rien, et me contenta de hocher la tête tout en observant cette espace si grandiose.\newline

Il me remit en route en me disant qu'on allait être en retard. On pris à droite où l'on monta une rampe. Il y avais foule, et ce n'étais sûrement pas le bas peuple qui était là. On se faufila parmi la masse, qui semblait inquiet, angoissé. On arriva dans une grande ouverture gardé par des soldats. Ce n'était pas les Séraphins, leur uniforme était rouge, c'était la garde du ministère. Yashu me tira dans les gradins qui surplombaient cette grande ouverture ovale, tapissé au sol d'une immense carte du royaume de Kryte en or massif.\newline

Les nobles, et autres curieux, prirent leurs places dans les gradins. Je demandai à Yashu ce qu'il voulait me montrer, il m'intima de me taire, ça allais commencer.\newline

« Mesdames et messieurs, le Ministre Ailoda ! »\newline

L'assemblé se tut, et les derniers debout prirent une place dans les gradins. La voix annonça l'arrivé des neufs principaux ministres de la Kryte, qui prirent chacun leur place dans l'assise du premier rang qui leur était dédié. La foule bavardait encore dans les gradins, on entendait des légers chuchotement venant de tous les coins de amphithéâtres pendant les annonces.\newline

« Mesdames et messieurs, le Ministre Légat, Caudecus Le Sage ! »\newline

Les chuchotis s'interrompirent brusquement. Le dernier ministre arriva, il était accompagné de deux gardes du ministère. Il était plutôt vieux, ses rides se faisaient facilement remarqué, et sa calvitie lui donnait une bonne soixantaine d'années. Son bouc, coloré de blanc, et très fièrement peigné, faisait pensé à un vieil érudit ayant passé sa vie à étudier, loin de tout, dans une bibliothèque sombre et sinistre. Tout dans cet homme semblait faire de lui le parfait bougre pour porter son titre de « sage » et imposer le respect du plus grand nombre. Surtout sa démarche, droite et confiante à laquelle on reconnait les plus grands dirigeants.\newline

Je n'avais qu'une hâte désormais. Entendre ce que cet homme avait à nous dire, et torturer de questions Yashu sur le pourquoi du comment il m'avait trainé ici.

\chapter{Le discours d'un Roi}
Caudecus pris lui aussi sa place dans les gradins. L'atmosphère était tendu. Seuls quelques petits chuchotis filtraient dans le silence solennel de l'assemblé.\newline

Une partie des ministres annoncés étaient assis d'un côté, l'autre morceau étant assis à l'extrémité de l'hémicycle. Ils se faisaient face comme de véritables rivaux. Tous n'attendaient qu'une chose, que les débats commencent. Mais on attendais quelqu'un, un dernier intervenant. L'homme le plus influent de la Kryte.\newline

Le capitaine Thackeray arriva finalement en compagnie de son fidèle lieutenant, Groban. Les deux hommes étaient en uniforme et contrastait magistralement avec le reste de l'assemblée, qui portait, soit des habits nobles, soit un uniforme de la garde du ministère.\newline

Le capitaine était présent aux débats ministériel pour une seule raison : c'était le champion de la reine. Et en tant que champion de la reine, il se devait de défendre la politique royal devant les Ministres, qui représentaient le peuple. Une centaine de représentants étaient élus par le peuple pour traiter les affaires du royaume. Parmi eux, un ministre légat était élu, c'était le chef des Ministres, chargé de nommé avec lui 8 autres ministres qui composerait finalement le gouvernement Krytien.\newline

Malgré cette organisation plutôt démocratique, la Reine possédait quoiqu'il arrive le dernier mot, et pouvait agir en contradiction de son gouvernement. Mais si celui-ci avait de quoi discréditer la royauté, les rôles pouvaient s'inverser en fonction du contexte, de la situation, mais aussi de la capacité et de la volonté du gouvernement à entreprendre une tel action.\newline

Un ministre pris la parole, c'était le doyen. C'est lui qui allait animer la séance, conformément à la tradition.\newline

« Mesdames et messieurs les ministres, mesdames et messieurs les visiteurs, cette séance exceptionnelle aura pour sujet principal les affaires étrangères Krytiennes. L'armée Centaure des Taminis a augmenté son nombre d'attaque sur nos ressortissants Krytiens, ils ont détruis en une semaine une dizaine de villages de nos concitoyens. La Reine, malgré une demande du gouvernement de mr Caudecus, a refusée d'envoyer plus de soldats au front, les préférant en ville afin de traquer la criminalité.

Mr le légat, dans une question écrite au capitaine Thackeray demande explications. Il interviendra en premier. La parole sera ensuite au gouvernement, qui pourra réagir aux déclarations de la royauté. Capitaine, la parole est à vous. »\newline

Thackeray s'avança devant l'auditoire, il avait la même démarche que Caudecus. De celle qui inspire à la fois respect, crainte et écoute. Il se tenait fermement debout devant la foule attentive, une main sur le pommeau de son épée, l'autre le long du corps.\newline

« La Kryte viens de vivre des journées grave, si sérieuse, que la Reine Jennah s'est longuement interrogé sur la tenue de ses engagements en partant en visite officiel à la Citadelle Noir. Après réflexion, elle a jugée que la situation Krytienne exigeait que les relations diplomatiques avec les Charrs devaient être tenus, conformément à ce qui était prévu.

La Reine m'a donné les pouvoirs qui sont prévus par les lois Krytiennes qui me permettent de prendre toutes dispositions nécessaire si la royauté venait à être indisponible et que la situation exigeant une action d'urgence. Quant à la Reine, elle s'expliquera publiquement quand elle sera rentrée de son voyage diplomatique.

J'ajoute que ces drames, ne sont pas dus à des défauts dans l'armée de Kryte. En effet les ordres ont été entendus, et mis en oeuvre dans les plus courts délais.

Grâce à eux, et à leur contre attaque rapide, ils ont permis de faire en sorte que le bilan de ces nuits ne soit pas trop lourd.

Mais en plus des armées centaures, il faut préciser qu'il y avait avec eux, des bandits, équipés de matériel adaptés aux pillages, et je ne pense non seulement qu'ils visent à créer la subversion dans toute la Kryte, mais qu'ils visent aussi à instaurer le trouble ici, au Promontoire Divin, au moment même où nous négocions la paix avec nos ennemis de toujours  »\newline

Une partie de l'auditoire applaudit à ce moment précis. L'autre moitié regardait, et attendait désespérément le moment de répondre.\newline

Thackeray continua une fois les applaudissements terminés.\newline

« Nous aurons à nous préoccuper de ces vauriens, afin qu'ils ne puissent nuire ni au peuple humain, ni à la royauté.

On me permettra de rappeler, sans aucune forfanterie, les efforts que nous avons faits dans la gestion de cette crise.

Le nombre de soldats Séraphins a doublé en moins de dix ans. Nous avons mis en place des organisations coûteuses, pour l'accueil des réfugiés au promontoire ainsi que pour les manoeuvres de notre armée hors de la capitale. Multipliant ainsi par quatre le budget alloué la défense Krytienne et à la gestion de ses conséquences.

Il ne peut y avoir que de solutions dans le travail d'équipe et l'investissement de tous. Rien ne serait plus illusoire que de penser que les évènements que nous avons vécu sont une flambée sans lendemains. »\newline

La même moitié d'auditoire applaudit à nouveau. Le capitaine avait finit son discours, probablement soigneusement préparé par un érudit de la cour royal. Caudecus, lui, avait pris des notes, il était prêt à répondre.\newline

Thackeray repartit s'assoir. Caudecus, lui, se leva, et monta à la tribune. Il avait préparé son réquisitoire.\newline

« Nous vous écoutions attentivement, Mr Thackeray et nous nous disions comme ce serait bien qu'il y aurait au pouvoir des gens capable de remplir la tache qui les définit.

Il y a de cela des années que votre régime est en place, mais nous pouvons le dire maintenant, précisément, car nous avons débattu des semaines durant, au sein même de cette enceinte, et que nous avions la parole de la Reine elle même, que la Kryte serait protégé.

On peut le dire, capitaine, qu'avez vous fais du royaume ? Où est la responsabilité ? Qui était responsable au cours de ses semaines ?

Alors, il semble, qu'a l'issu d'un des conseils des ministres, la Reine aurait déclarée qu'elle souhaitait chasser la racaille urbaine plutôt que de défendre nos terres !

Qui est responsable au cours de ces derniers jours ? La Reine ? Bien sûr, vous ne pouvez pas répondre à ma question.

Alors, répondez nous Capitaine, qu'avez vous fais du royaume ? Qui était responsable ?

Ce qui est sûr, c'est qu'il existait un pouvoir responsable, ici au Promontoire. Qu'avez vous fais pour éviter ces massacres capitaine ?

Votre Reine vous a-t-elle mesuré le temps ? Vous a-t-elle mesuré les pouvoirs ? Et même les pleins pouvoirs ? Afin de prévoir, de diriger, de gouverner !

Qu'avez vous fais de notre royaume Capitaine ?  »\newline

Cette fois, ce fut à l'autre moitié d'applaudir.\newline

« Qui donc était responsable ? Où est votre responsabilité ? Et peut être êtes vous devant nous, car il le faut bien, mais qu'il ne faut pas avouer le point où vous en êtes ! Et notre cher Capitaine, de retour des champs de batailles, assume ! »\newline

Cette dernière phrase avait provoqué l'émoi dans l'auditoire. Certains applaudissaient frénétiquement, les autres huaient et tapaient sur les bancs en signe de mécontentement du discours du Légat. L'animateur de séance leva la voix pour demander le silence.\newline

« Au demeurant, qu'arrivera-t-il dans les jours prochains ? Qu'avez vous fait de l'autorité de l'état ? Vous avez laissé un peuple s'abaisser, contre une simple menace de sauvages ! Au moment même où nous négocions la paix avec nos ennemis de toujours, qu'on aurai tous préféré voir six pieds sous terre !

Ainsi, Capitaine, nous pensons qu'il est temps que l'hypocrisie cesse, et que la royauté écoute enfin le gouvernement ministériel, et qu'il est temps que la monarchie s'en aille !  »\newline

Caudecus finit là son discours. La foule était hystérique, certains acclamaient les dires du Légat, d'autres le huait. Si la garde ministériel n'était pas là, tout ceci aurait finit en un bain de sang, et je ne comprenais même pas pour quoi.\newline

D'un côté je sentis un homme qui avais tous fais pour éviter le pire, de l'autre quelqu'un qui disait que ça aurait pu être le cas si on y aurais mis les moyens. Je ne savais que penser. Yashu allait devoir m'éclairer et vite.\newline

Le doyen sonna la fin de la séance. L'ambiance était telle que la discussion n'était plus possible. La foule se dispersa et chacun donna dans son groupe son avis sur la question. Les principaux ministres nommés au début de l'algarade semblait même pas être d'accord entre eux. J'y comprenais rien.\newline

Yashu me fit signe de le suivre. On commença à s'éloigner de l'amphithéâtre. Je vis que le soleil commençait à décliner dans le ciel, et que ces beaux discours avaient duré bien quelques heures.\newline

Je ne savais toujours pas où j'allais dormir, et je demanda à Yashu les explications qu'il me manquait tant. Il me dis seulement de prendre note de ce qu'il s'était passé aujourd'hui, et que pour le moment, il voulait me présenter à ce qu'il appellait des « amis ».\newline

Selon lui, si je le suivais, je pourrai avoir de quoi vivre tranquillement, un travail, un coin où dormir à l'abri des pluies torrentielles de la saison. Ce qu'il me proposait était bien trop beau, mais pourquoi moi ?\newline

Je décida finalement de le suivre. Je ne savais pas pourquoi mais j'étais prête à lui faire confiance, je sentais que je n'allais probablement pas le regretter.\newline

\chapter{La Guilde des Mercenaires}
Nous étions en plein après midi, il faisait chaud. Le soleil resplendissait de mille feux au dessus de la ville, ce qui la rendait magnifique. Le débat des ministres avais facilement duré plusieurs heures, entre les différents protocoles et les discours des deux personnages les plus influents de la ville, et même du royaume.\newline

Avant de rencontrer les amis de Yashu, je voulais avoir ses explications, pourquoi tout ce cinéma, pourquoi moi ? Il me répondit tandis que nous nous baladions dans les jardins royaux.\newline

Yashu travaillait pour  une guilde. En effet, la Tyrie comptait trois ordres indépendantes les unes des autres et indépendantes de toute autorité des différentes nations. Mais les évènements récents avaient conduis à la formation de plus petites structures, semblables au ordres, nommés les guildes, afin de répondre à un seul objectif précis.\newline

Au Promontoire Divin, la guerre contre les Centaures avaient poussé les réfugiés à créer une guilde dédié au combat de cet envahisseur. En effet, les Séraphins n'avaient pas les moyens de former et équiper tous les réfugiés qui demandaient vengeance, d'où la création de ces groupes, autonome de l'action royale.\newline

Yashu était donc dans cette guilde, et ils recherchaient sans cesse de nouvelles recrues afin de gonfler leur rang. Cependant, leurs moyens limités ne pouvait leur permettre de former une armée. Ils se contentèrent donc, en groupe, d'éliminer des menaces précises, mais les besoins de financement les avait éloigné de leur objectif initial et était devenu une sorte de groupe de mercenaires.\newline

Leur chef était un certain Mathew, il avait fondé le groupe il y a de cela à peine un mois. C'était un ancien veilleur, qui aimait le combat, et qui en plus était doué pour ça. Il n'avait aucune envie de devenir simple soldat ou garde, il voulait être dans le feu de l'action, au coeur des décisions, s'investir pour un objectif qui serait le sien, et celui de son groupe.\newline

En tant qu'ancien Maître de Guerre des veilleurs, Mathew avait acquis une certaine renommée, mais aussi une certaine richesse, qui lui permit de se procurer une grande maison dans le quartier populaire ouest du Promontoire. Yashu me fit signe de le suivre, on s'y rendait, et continuerai ses explications en chemin.\newline

« Une fois le bâtiment à disposition, on avait tout pour pouvoir s'entrainer, dormir, et entreposer notre équipement. Bien sûr ce n'est certainement pas le grand luxe, mais on avait tout pour démarrer.

- Comment ça, « on » ? lui demandai-je

- Mathew n'as pas entrepris son projet seul, on étais trois à le suivre. Tous d'ancien veilleurs. Moi je suis un ancien croisé, souvent sous les ordres de Mathew, mais il y as aussi un ancien Tacticien, Zane, ainsi qu'une ancienne Tireuse d'élite, Susan. C'est impossible de commencer un tel projet sans un minimum de personnes expérimenté vois-tu, ce serait du pur suicide »\newline

Une vie de mercenaires. Ça me conviendrait bien. Si il y avait bien un endroit où je pourrai trouver ma vengeance un jour, c'était là.\newline

Yashu n'avait pas répondu à la question fatidique, pourquoi moi ?\newline

« Je t'ai vu arrivé au Promontoire, en compagnie de ce pauvre paysan. Il ramène constamment des réfugiés, et la raison est simple : il vit assez loin de la capitale, mais vient ici assez régulièrement. »\newline

Sa recherche constante de chair fraiche à entrainer faisait de lui quelqu'un d'étonnamment au courant des avancées de la guerre, mais aussi de toutes les arrivés de réfugiés. C'était d'ailleurs pour ça qu'il se rendait régulièrement aux conseils des ministres.\newline

« Quand je t'ai vu, j'ai tout de suite compris que tu étais la seule survivante des massacres à la frontière du Royaume. Si tu n'aurais aucun potentiel pour nous, tu n'aurais jamais réussi à t'enfuir de cette bataille »\newline

La pure coïncidence, le hasard ainsi que la chance m'avait permit d'être encore en vie, d'avoir atterri ici, et en plus de croiser sa route. J'étais incapable de savoir quoi penser de tout ça, et encore moins de savoir quoi lui répondre. Je fus muet comme une tombe.\newline

On arriva au quartier populaire ouest, là où était situé le quartier général de la guilde. On repassa par la place où était accueilli ou recensé les exilés, mais on passa tout droit ce coup ci. « Nous y voilà » dit Yashu en montrant d'un hochement de tête la demeure.\newline

Il m'expliqua que le règlement n'était pas strict mais qu'en tant que nouvelle recrue, je devrais m'entrainer souvent, en commençant aujourd'hui, et que tant que je serai pas au minimum au point, je ne pourrai pas aller à la taverne situé au bout de la rue. C'était le lieu des réunions stratégiques selon lui.\newline

On entra et fit les présentations. Seul Mathew était présent, aucun signe de Susan ou même de Zane. Mathew était un homme grand, sa musculature montrait clairement son appartenance à une armée d'élite, et son visage était impeccable, signe de toute la discipline militaire. En effet, son crâne était rasé de prêt, et c'était pareil pour sa pilosité qui pourrait être Caledon si l'entretien n'étais pas au rendez vous.\newline

Étonnamment, il ne semblait pas autoritaire, il avait un grand sourire en voyant qu'une nouvelle recrue venait d'entrer dans ce qui aurait pu être son salon - en effet la pièce où il était semblait être comme un grand hall, au centre duquel se dressait une grande table sur laquelle était amassé quelques paperasses à la place du chef - Il semblait même être très à l'écoute de ses Hommes, mais au timbre de sa voix, je compris de suite qu'il était largement capable de donner des ordres et des réprimandes si c'était nécessaire.\newline

« Bon, nos deux amis ne vont pas tarder à rentrer d'expédition, passons donc aux choses sérieuses ! dit il avec un grand sourire narquois, qui l'espace d'un instant me fit peur.

- Heu, c'est quoi les choses sérieuses ? »\newline

J'étais incertaine de moi, je doutais de mes capacités, je ne savais pas ce qu'était les « choses sérieuses » dont il me parlait.\newline

Il m'intima l'ordre de le suivre. On alla vers une des portes qui débouchait vers une autre salle situé au rez de chaussée. C'était une grande salle, probablement la même taille que le hall, au centre de laquelle était suspendu plusieurs mannequins d'entrainement manufacturé à partir de paille dans un vulgaire sac de jutes.\newline

Tout autour, sur les murs, il y avait moult armes différentes : épées, dagues, boucliers, arcs, flèches, masses, bâtons. De quoi armer une bonne quinzaine de personnes.\newline

« Comment imagine tu te battre Jèssi ? me demanda Mathew »

Je me questionnais. Il était sérieux ? Où voulait il en venir ?

Yashu me fit un signe de la tête, signe que je devais avoir confiance et répondre franchement à la question.

« Je veux être au plus près de mes ennemis, sentir leur sang quand ils mourront et pouvoir trancher la tête de ceux qui ont détruit ma vie »

Un grand sourire illuminait le visage de mes deux bienfaiteurs.

« Une guerrière hein ? On va pouvoir bien briser des crânes avec ça » dit en riant le chef.\newline

Il sortit une épée à deux mains de la réserve d'armes, elle n'était pas aiguisé et ne pourrait même pas trancher du beurre, et encore moins un centaure.\newline

L'arme était grande, au moins un bon mètre, c'était une vulgaire lame en fer, la garde était en lanière de cuir. On aurait dit une vieille arme de paysans révolté s'en allant faire la révolution tout seul contre la reine.\newline

« Prends ça la bleu ! Et tapes sur le mannequin ! » m'ordonna Mathew.\newline

Je pris la lame, j'étais à peine capable de la soulever. Mais je devais y arriver, je la souleva donc, et tapait de toutes mes forces le mannequin. Cette lame faisait bien la dizaine de kilos. Sans doute était-elle plus lourde que la normal pour les entrainements.\newline

« Bon, t'es engagé. Le programme chez nous est assez simple, repas le matin, entrainement ensuite, puis repas, puis ré-entrainement ou alors mission, puis repas et enfin dodo, compris ?

- Et les réunions stratégiques, c'est quand ? Yashu m'en as parlé, et je veux y participer ! »\newline

Le trait d'humour avait tout de suite été compris par le chef. Et il ne put s'empêcher de me charrier.

« Quand tu seras capable de porter et de taper avec une vrai épée à deux mains, gamine ! »\newline

Je m'entraina toute la fin de journée, puis vint le moment du repas. Je rencontrai enfin les deux autres membres du groupe, qui me souhaitèrent gentiment la bienvenue.\newline

Zane était un homme plutôt petit, mais il était moyennement large, et équipé de son armure, il était plutôt impressionnant à voir. Il avait tout du cliché du barbare, malgré sa voix plutôt douce et attendrissante.\newline

Susan, elle, était mince, grande d'un bon mètre quatre-vingt. Elle était plus grande que Zane et cela faisait d'eux un duo complètement improbable. Elle était plutôt âgé, avait passé sa vie à tirer à l'arc, et avais maîtrisé son plus grand point faible : ses longs cheveux blonds retombant devant ses yeux et qui pouvaient gêner sa visé.\newline

On se mit à table, Zane et Susan se mirent d'un côté, moi et Yashu en face, le chef en bout. On discutta un long moment. D'abord, le duo improbable raconta son après midi. Ils avaient étés appelés en renfort par les Séraphins pour traquer une bête qui menaçaient le bétail du coin. Ce genre de travail était mal payé, mais permettait au groupe de se faire connaitre et d'espérer mieux.\newline

Ensuite, chacun discuta dans son coin, le chef était plongé dans ses papiers, Zane et Susan parlait chiffon, et Yashu me demanda de lui raconter mon histoire, ce fut sans grande envie que je la lui raconta.\newline

La fin de la soirée fut pour moi l'occasion de découvrir les dortoirs à l'étage, auquel on accédait par la porte à l'opposé de la salle d'entrainement. Pour le moment, la demeure avait gardé ses anciennes chambres individuelles, mais il était prévu de réaménager le tout pour faire un véritable dortoir collectif.\newline

Tant mieux pour moi, je voulais être seule. On me guida à ma chambre. Une fois seule, j'étais là, planté comme un piquet au milieu de la pièce. Elle était étroite, un lit une place dans un coin de la chambre et une fenêtre guère plus grande qu'une meurtrière qui donnait sur le jardin de la maison où était mis à disposition cibles et flèches pour l'entrainement à l'arc.\newline

Je tomba de fatigue sur le lit, mais je fus incapable de dormir. Tant de choses s'était déroulé si vite, je ressentais une foule de sentiments à la fois. Ces centaures, ma fuite, ce paysan, Yashu, les discours, cet entrainement. J'étais à la fois triste, enragée, pleine d'espoirs, et de doutes. Et peut être même amoureuse.\newline

Je mis mon visage dans l'oreiller, et je pleurai.\newline




\end{document}