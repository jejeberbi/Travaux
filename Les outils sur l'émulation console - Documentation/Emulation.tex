\documentclass{report}

\usepackage[latin1]{inputenc}
\usepackage[T1]{fontenc}
\usepackage[francais]{babel}
\usepackage[top=2cm, bottom=2cm, left=4cm, right=4cm]{geometry}
%\usepackage{charter}
\pagestyle{plain}
\author{Jejeberbi : www.jejeberbi.eu}
\date{Janvier 2012}
\title{Les outils de l'�mulation console}

\begin{document}

\maketitle
\tableofcontents

\chapter*{Pr�ambule}

\section{Introduction}
	Pouvoir rejouer � des jeux anciens, qui ont marquer notre jeunesse, est quelque chose de sublime. Cependant, ne disposant en g�n�ral plus des consoles de jeux, il devient d�r d'y rejouer. Mais des alternatives existent, ce document a pour but de pr�senter les divers outils existants afin de (re--)jouer � des jeux d'antan. Attention, ce document ne traite pas de comment faire un �mulateurs et des principes qui sont autour.

\section{\'A qui est destinn�e ce document ?}
	Ce document est destinn� � n'importe qui souhaitant en apprendre plus sur les outils de l'�mulation, et qui souhaite jouer � quelques jeux. Un �mulateur n'est pas un outil pour jouer massivement � des jeux, et ne sert absolument pas � contourner l'achat des jeux. Il est d'ailleurs impossible d'�muler les consoles r�cente. Si vous souhaitiez donc �muler le dernier jeux de guerre � la mode pour ne pas payer la console, vous �tes au mauvais endroit.

\section{Les outils pr�sent�s}
	Je vais donc pr�senter plusieurs �mulateurs et faire le tour des options qu'ils fournissent. Utiliser un �mulateur n'a rien de d�r en soit, mais conna�tre les options qui l'entoure est tr�s int�rressant et peux augmenter le confort de jeux si l'on sait s'en servir. Ce document n'est pas exaustif concernant ces options (~et oui je ne connais pas tout~) mais ce que je vais expliquer ici vous donnera, je pense, d'assez bonnes basses pour que vous puissiez vous d�brouiller tout seul. \newline
	
	Les �mulateurs pr�sent�s ne sont pas des �mulateurs de tr�s anciennes consoles, mais plut�t de machines encore assez r�cente dont on peux encore trouver des jeux dans les commerce. Bon, tr�ve de bavardage, voil� la liste de ce que nous allons voir~:
	\begin{itemize}
\item \'Emulateur Playstation 1~: PSX~;
\item \'Emulateur Playstation 2~: PCSX2~;
\item \'Emulateur Game Boy Advance, Nintendo DS, avec pleins d'autres outils sympatiques~: WinDS PRO~;
	\end{itemize}
Cette liste est certe un peu petite, mais les trois points de vue � propos des options abord�s dans ce document vous permettra d'avoir quelques connaissances pour chercher, tester, et configurer d'autres �mulateurs. \newline

\chapter{L'emulateur PS1, PSX}
\section{Pr�sentation}
\begin{itemize}
	\item Nom~: PSX~;
	\item \'Emule la Playstation 2, la console de Sony, sorti autour des ann�es 2000.
	\item Caract�ristiques~: prend en charge le d�placement via Joystick et les cartes m�moires % Voir annexe A et l'utilisation de DS3 Tool.
\end{itemize}

\end{document}