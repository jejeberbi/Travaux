\documentclass[t]{beamer}
\usepackage[latin1]{inputenc}
\usepackage[francais]{babel}
\usepackage[cyr]{aeguill}
\usepackage{setspace}
\usetheme{Hannover}

\date{2012}
\author{J�r�my B.}
\title{Les dangers d'Internet}

\begin{document}

\frame{\titlepage}


%Table des mati�res
\begin{frame}
\tableofcontents[hideallsubsections]
\end{frame}
% Fin table des mati�res

\section{Qu'est-ce que Internet ?} % On explique ce qu'est le r�seau...

\subsection{Niveau mat�riel} 
\begin{frame}
{\Large Comment �a marche au niveau mat�riel...} % ... Niveau matos ...
\pause

Il faut savoir que le r�seau est~:
\pause

\begin{itemize}
	\item � commutation de paquet, contrairement au t�l�phone qui est � commutation de circuit ;
	\pause

	\item a-centr�, il n'a pas de centre, contrairement au minitel ;
	\pause

	\item b�te � manger du foin, ce qui est au milieu passe son temps � raconter sa vie.
\end{itemize}
\pause

Conclusion~: Le "centre" compos�e de machines d�biles se racontent leurs vie pour savoir o� est qui et pour ensuite envoyer les donn�es (paquets) qui serviront � l'intelligence, en p�riph�rie du r�seau.
\end{frame}





\subsection{Niveau application}
\begin{frame}
{\Large ... et au niveau applicatif}
\pause

Au niveau applicatif, il faut distinguer les:
\begin{itemize}
	\item protocoles ;
	\pause
	
	\item les applications utilisant un protocole.
\end{itemize}
\pause

Il faut aussi savoir que~: chaque application communicant sur le r�seau a besoin d'un port o� envoyer et recevoir ses paquets. Il y en a plus de 5000 sur un ordinateur.
\end{frame}



\begin{frame}
{\Large Les protocoles}
\pause

Les protocoles sont ce qui permet d'�changer d'une certaine fa�on des paquets entre ordinateur~:
\pause

\begin{itemize}
	\item IP~: Internet Protocol, c'est un protocol basique d'adressage de machine ;
	\pause

	\item TCP ; UDP~: pour le transport basique de paquet ;
	\pause
	
	\item SMTP ; IMAP~: pour l'envoie et la r�c�ption de mail (port 143, 25) ;
	\pause
	
	\item FTP~: pour l'envoi de fichier (port 21) ;
	\pause
	
	\item DNS~: pour lier un nom de domaine � une IP (port 53) ;
	\pause
	
	\item P2P~: peer to peer, pour l'�change de fichier (port al�atoire, variant selon le client P2P) ;
	\item etc...
	\end{itemize}
	
\end{frame}


\begin{frame}
{\Large Les applications}
Les applications sont les logiciels qui se servent des protocoles et du mat�riel pour offrir des services.

Comme~:
\pause
\begin{itemize}
	\item Le mail ;
	\item MSN ;
	\item Facebook ;
	\item Jabber ;
	\item Ares, uTorrent, Emule etc...
\end{itemize}
\end{frame}


\subsection{Conclusion}
\begin{frame}
{\Large Conclusion premi�re partie}
\begin{itemize}

	\item La partie mat�riel a aussi besoin de la partie applicatif (procole).
\pause


	\item Le mat�riel ne peut communiquer si il n'existe pas un protocole qui le permette (en l'occurence ici, le BGP pour les routeurs).
\pause


	\item Les protocoles se situent � diff�rents "niveaux" les un des autres. Ce qui explique pourquoi certains n'ont pas de ports par d�faut.
\pause


	\item Cette caract�ristique qui forme Internet s'appelle le {\emph Mod�le OSI}.
\end{itemize}
\end{frame}







\section{Les d�rives}
\begin{frame}
{\Large Les d�rives}
Voici maintenant les sois-disantes d�rives de l'Internet et 






\end{document}